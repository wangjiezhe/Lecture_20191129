\documentclass[a4paper]{exam}

\usepackage{everyday}

\title{\huge \heiti \color{dblue} 每日一题(30)\\\pgfornament[scale=0.4]{84}}
\author{}
\date{}

\begin{document}

\maketitle

\thispagestyle{headandfoot}

% \qformat{\Large \textbf{题 \thequestion。}\hfill}

\begin{questions}

\setcounter{question}{29}

\question
% Origin: http://lanqi.org/everyday/29019/
设 $S(x)$ 表示自然数 $x$ 的数字和,求方程 $x+S(x)+S\left(S\left(x\right)\right)=2019$ 的所有自然数解。

\begin{solution}[\stretch{1}]
    易知 $x$ 为四位数,且 $1900<x<2100$.

    \begin{enumerate}[label={(\arabic*)}]
        \item $x=\overline{19ab}$,此时方程即 \[(1900+10a+b)+(1+9+a+b)+S(10+a+b)=2019,\]
        注意到 $a+b \leqslant 18$,于是 $S(10+a+b)=S(a+b)+1$,
        故 \[11a+2b+S(a+b)=108\]
        \begin{enumerate}[label={(\roman*)}]
            \item 若 $a+b<10$,则有 $S(a+b)=a+b$,化简得 $4a+b=36$,解得 $(a,b)=(4,8) \text{~或~} (7,8)$,均不满足要求;
            \item 若 $a+b\geqslant 10$,则有 $S(a+b)=a+b-9$,化简得 $4a+b=39$,解得 $(a,b)=(8,7) \text{或} (9,3)$。
        \end{enumerate}
        所以 $x=1987,1993$ 符合题意。
        \item $x=\overline{20ab}$,此时方程即\[(2000+10a+b)+(2+0+a+b)+S(2+a+b)=2019,\]即\[11a+2b+S(2+a+b)=17\]
        \begin{enumerate}[label={\roman*)}]
            \item 若 $a+b<8$,则有 $S(a+b+2)=a+b+2$,化简得 $4a+b=5$,解得 $(a,b)=(0,5) \text{或} (1,1)$;
            \item 若 $a+b\geqslant 8$,则有 $S(a+b+2)=a+b-7$,化简得 $4a+b=8$,解得 $(a,b)=(0,8)$。
        \end{enumerate}
        所以 $x=2005,2008,2011$ 符合题意。
    \end{enumerate}

    综上所述,所求方程的自然数解为 $1987,1993,2005,2008,2011$。

    \textcolor{red}{\textbf{另解:}}
    注意到 $x \equiv S(x) \equiv S\left(S\left(x\right)\right) \pmod{9}$,可知 $x \equiv 1 \pmod{3}$。
    另外,
    \[
        \begin{aligned}
            0<&S(x)\leqslant 1+9+9+9=28, \\
            0<&S\left(S\left(x\right)\right)\leqslant 1+9=10,
        \end{aligned}
    \]
    故 $1981\leqslant x < 2019$。于是 $x$ 的所有可能值为:
    \[1981,1984,1987,1990,1993,1996,1999,2002,2005,2008,2011,2014,2017,\]
    之后逐一验证即可。
\end{solution}

\end{questions}

\end{document}
