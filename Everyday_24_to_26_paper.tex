\documentclass[a4paper]{exam}
\usepackage[UTF8,linespread=1.5]{ctex}

\usepackage{fourier}

\catcode`。=\active
\def。{.}

\usepackage{mathtools}

\renewcommand{\solutiontitle}{\noindent\textbf{解答:}\par\noindent}


\title{每日一题(24-26)}
\author{}
\date{}

\begin{document}

\maketitle

\thispagestyle{empty}

\begin{questions}

\question

一个自然数的首位数字是\(4\),将其首位数字移动至末尾之后,得到的新数是原来的\(\dfrac{1}{4}\),求满足条件的最小自然数。

\begin{solution}[\stretch{1}]
    设 \(n = 4 \times 10^k + m\),\(n' = 10 m + 4\),其中\(k\)和\(m\)都是自然数,\(k \ge 2\),\(m < 10^k\),\\
    则 \(n = 4n'\),即 \(4 \times 10^k + m = 4 \times \left(10  m + 4\right)\)。\

    化简可得 \(39 m = 4 \left(10^k - 4\right)\)。\

    因为 \(\left(39,4\right) = 1\),所以\(39 \mid \left(10^k - 4\right)\),即 \(10^k \equiv 4 \pmod{39}\)。\

    因为 \(10^k \equiv 1^k \equiv 1 \equiv 4 \pmod 3\) 恒成立,所以只需考虑 \(10^k \equiv 4 \pmod{13}\)。\

    经验证,\(10^2 \equiv 9 \pmod{13}\),\(10^3 \equiv 12 \pmod{13}\),\(10^4 \equiv 3 \pmod{13}\),\(10^5 \equiv 4 \pmod{13}\),

    所以 \(k\) 最小是 \(5\)。\

    此时 \[m = \frac{10^5-1}{39} = 10256,\; n = 410256.\]
\end{solution}

\question

设\(p\)是奇质数,如果\(1+\dfrac{1}{2}+\cdots+\dfrac{1}{p-1}=\dfrac{m}{n}\),其中\(m\)、\(n\)是互质的正整数,求证:\(p \mid m\)。

\begin{solution}[\stretch{1}]
    \[
        \begin{aligned}
            \frac{m}{n} &= \left(1+\frac{1}{p-1}\right) + \left(\frac{1}{2}+\frac{1}{p-2}\right) + \cdots + \left(\frac{1}{\frac{p-1}{2}}+\frac{1}{\frac{p+1}{2}}\right) \\
            &= \frac{p}{1\left(p-1\right)} + \frac{p}{2\left(p-2\right)}+ \cdots + \frac{p}{\frac{\left(p-1\right)}{2} \cdot \frac{\left(p+1\right)}{2}} \\
            &= p \cdot \left(\frac{1}{1\left(p-1\right)} + \frac{1}{2\left(p-2\right)}+ \cdots + \frac{1}{\frac{\left(p-1\right)}{2} \cdot \frac{\left(p+1\right)}{2}}\right) \\
            &= p \cdot \frac{M}{N}
        \end{aligned}
    \]

    其中\(\left(M,N\right)=1\)。

    注意到\(N = \left[1,2,\cdots,p-1\right]\),且\(p\)为质数,则\(\left(p,N\right)=1\)。

    所以\(m = pM\),\(n = N\),因此\(p \mid m\)。
\end{solution}

% \newpage

\question

已知\(\overline{abc}\)是\(37\)的倍数,求证:\(\overline{bca}\)也是\(37\)的倍数。

\begin{solution}[\stretch{1}]
    \[
        \begin{aligned}
            \overline{bca} &= 100b+10c+a \\
            &= 10 \times \left(100a+10b+c\right) - 999a \\
            &= 10 \overline{abc} - 999a
        \end{aligned}
    \]

    因为 \(37 \mid \overline{abc}\),\(37 \mid 999\),所以 \(37 \mid \overline{bca}\)。
\end{solution}

\end{questions}

\end{document}
