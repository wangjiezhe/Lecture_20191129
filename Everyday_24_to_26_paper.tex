\documentclass[a4paper]{exam}
\usepackage[UTF8]{ctex}

% \usepackage{geometry}
% \geometry{a4paper,scale=0.8}

\usepackage{fourier}
% \usepackage{mathpazo}

\catcode`。=\active
\def。{.}

\usepackage{mathtools}

% \usepackage{setspace}
% \setstretch{1.3}

\title{每日一题(24-26)}
\author{}
\date{}

\begin{document}

\maketitle

\pagestyle{empty}
\thispagestyle{empty}

\begin{questions}

\question

一个自然数的首位数字是\(4\),将其首位数字移动至末尾之后,得到的新数是原来的\(\dfrac{1}{4}\),求满足条件的最小自然数。

\vspace*{\stretch{1}}

\question

设\(p\)是奇质数,如果\(1+\dfrac{1}{2}+\cdots+\dfrac{1}{p-1}=\dfrac{m}{n}\),其中\(m\)、\(n\)是互质的正整数,求证:\(p \mid m\)。

\vspace*{\stretch{1}}

\question

已知\(\overline{abc}\)是\(37\)的倍数,求证:\(\overline{bca}\)也是\(37\)的倍数。

\vspace*{\stretch{1}}

\end{questions}

\end{document}
