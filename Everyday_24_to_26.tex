\documentclass[UTF8]{ctexbeamer}

\usetheme{Madrid}
\usecolortheme{whale}
\usefonttheme{professionalfonts}
% \usefonttheme{serif}

% \usepackage{fontspec}
% \setCJKmainfont{Source Han Serif SC}
\setCJKsansfont{Source Han Sans SC}
% \setmainfont{TeX Gyre Bonum}

\usepackage{fourier}
% \usepackage{mathpazo}

\catcode`。=\active
\def。{.}

\usepackage{mathtools}

\usepackage{setspace}
\setstretch{1.3}

\newenvironment<>{question}[1][]{\begin{block}#2{问题}}{\end{block}}
\newenvironment<>{analysis}[1][]{\begin{exampleblock}#2{分析}}{\end{exampleblock}}
\newenvironment<>{solution}[1][]{\begin{alertblock}#2{解答}}{\end{alertblock}}

\author{王介哲}
\institute[]{优才教育}
\title{每日一题}
\subtitle{24-26}
\date[Nov 2019]{2019年11月29日}

\begin{document}

\frame{\titlepage}

\begin{frame}
    \frametitle{每日一题(24)}

    \begin{question}
        一个自然数的首位数字是\(4\),将其首位数字移动至末尾之后,得到的新数是原来的\(\dfrac{1}{4}\),求满足条件的最小自然数。
    \end{question}\pause

    \begin{analysis}
        \begin{itemize}[<+-| alert@+>]
            \item 我们可以设\(n\)是一个\(k\)位自然数,且首位数字是\(4\),则\(n = 4 \times 10^k + m\)。
            \item 于是,变换之后的数字为\(n' = 10 m + 4\)。
            \item 根据条件,我们有\(n = 4n'\)
            \item 接下来,解关于\(m\)和\(k\)的不定方程即可。
        \end{itemize}
    \end{analysis}
\end{frame}

\begin{frame}
    \frametitle{}

    \begin{solution}
        设 \(n = 4 \times 10^k + m\),\(n' = 10 m + 4\),其中\(k\)和\(m\)都是自然数,\(k \ge 2\),\(m < 10^k\),\\
        则 \(n = 4n'\),即 \(4 \times 10^k + m = 4 \times \left(10  m + 4\right)\)。\pause

        化简可得 \(39 m = 4 \left(10^k - 4\right)\)。\pause

        因为 \(\left(39,4\right) = 1\),所以\(39 \mid \left(10^k - 4\right)\),即 \(10^k \equiv 4 \pmod{39}\)。\pause

        因为 \(10^k \equiv 1^k \equiv 1 \equiv 4 \pmod 3\) 恒成立,所以只需考虑 \(10^k \equiv 4 \pmod{13}\)。\pause

        经验证,\(10^2 \equiv 9 \pmod{13}\),\(10^3 \equiv 12 \pmod{13}\),\(10^4 \equiv 3 \pmod{13}\),\(10^5 \equiv 4 \pmod{13}\),

        所以 \(k\) 最小是 \(5\)。\pause

        此时 \[m = \frac{10^5-1}{39} = 10256,\; n = 410256.\]
    \end{solution}

\end{frame}

\begin{frame}
    \frametitle{每日一题(25)}

    \begin{question}
        设\(p\)是奇质数,如果\(1+\dfrac{1}{2}+\cdots+\dfrac{1}{p-1}=\dfrac{m}{n}\),其中\(m\)、\(n\)是互质的正整数,求证:\(p \mid m\)。
    \end{question} \pause

    \begin{analysis}
        \begin{itemize}[<+-| alert@+>]
            \item 基本思路:左边通分,证明分子中含有质因子\(p\)。
            \item 如果直接全部通分的话,分子会变得复杂,无法判断它的因子。
            \item 考虑首位配对:\[1+\frac{1}{p-1}=\frac{p}{p-1},\; \frac{1}{2}+\frac{1}{p-2}=\frac{p}{2\left(p-2\right)},\;\cdots\]  发现分子均为\(p\)。
        \end{itemize}
    \end{analysis}

\end{frame}

\begin{frame}
    \frametitle{}

    \begin{solution}
        \[
            \begin{aligned}
                \frac{m}{n} &= \left(1+\frac{1}{p-1}\right) + \left(\frac{1}{2}+\frac{1}{p-2}\right) + \cdots + \left(\frac{1}{\frac{p-1}{2}}+\frac{1}{\frac{p+1}{2}}\right) \\
                &= \frac{p}{1\left(p-1\right)} + \frac{p}{2\left(p-2\right)}+ \cdots + \frac{p}{\frac{\left(p-1\right)}{2} \cdot \frac{\left(p+1\right)}{2}} \\ \pause
                &= p \cdot \left(\frac{1}{1\left(p-1\right)} + \frac{1}{2\left(p-2\right)}+ \cdots + \frac{1}{\frac{\left(p-1\right)}{2} \cdot \frac{\left(p+1\right)}{2}}\right) \\ \pause
                &= p \cdot \frac{M}{N}
            \end{aligned}
        \]

        其中\(\left(M,N\right)=1\)。 \pause

        注意到\(N = \left[1,2,\cdots,p-1\right]\),且\(p\)为质数,则\(\left(p,N\right)=1\)。 \pause

        所以\(m = pM\),\(n = N\),因此\(p \mid m\)。
    \end{solution}

\end{frame}

\begin{frame}
    \frametitle{每日一题(26)}

    \begin{question}
        已知\(\overline{abc}\)是\(37\)的倍数,求证:\(\overline{bca}\)也是\(37\)的倍数。
    \end{question} \pause

    \begin{analysis}
        \begin{itemize}[<+-| alert@+>]
            \item 考虑保证倍数关系不改变的变换:
            \begin{itemize}
                \item 乘以一个常数,或者除以一个与\(37\)互质的常数
                \item 加上(或减去)一个\(37\)的倍数
            \end{itemize}
            \item \(\overline{abc} = 100a+10b+c\),\(\overline{bca}=100b+10c+a\),如何将\(10b+c\)变成\(100b+10c\)?
            \item 注意:\(37 \mid 111 \mid 999\)
        \end{itemize}
    \end{analysis}

\end{frame}

\begin{frame}
    \frametitle{}

    \begin{solution}
        \[
            \begin{aligned}
                \overline{bca} &= 100b+10c+a \\
                &= 10 \times \left(100a+10b+c\right) - 999a \\
                &= 10 \overline{abc} - 999a
            \end{aligned}
        \] \pause

        因为 \(37 \mid \overline{abc}\),\(37 \mid 999\),所以 \(37 \mid \overline{bca}\)。
    \end{solution}

\end{frame}

\end{document}
