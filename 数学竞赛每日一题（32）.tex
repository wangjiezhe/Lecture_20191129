\documentclass[a4paper]{exam}
\usepackage[UTF8,linespread=1.5]{ctex}

\pagestyle{headandfoot}
\header{}{}{}
\footer{}{\small \quad 第~\thepage~页(共~\numpages~页)}{}

\usepackage{fourier}
% \usepackage{newtxmath}
% \usepackage{ebgaramond}

\makeatletter
\providecommand{\nmid}{} % a mock definition
\renewcommand\nmid{\mathrel{\mathpalette\thiel@nmid\relax}}
\newcommand{\thiel@nmid}[2]{%
  \ooalign{%
    \rotatebox[origin=c]{40}{$\m@th#1-\mkern-3.5mu$}\cr
    \hidewidth$\m@th#1|$\hidewidth\cr}%
}
\newcommand\wmid{\mathrel{\mathpalette\thiel@mid\relax}}
\newcommand{\thiel@mid}[2]{%
  \ooalign{%
    \hphantom{\rotatebox[origin=c]{40}{$\m@th#1-\mkern-3.5mu$}}\cr
    \hidewidth$\m@th#1|$\hidewidth\cr}%
}
\makeatother

\usepackage{mathtools}
% \usepackage{amsmath,amssymb,amsthm}
% \usepackage{unicode-math}
% \usepackage{mathabx}

\catcode`。=\active
\def。{.}

\usepackage{pgfornament}
\usepackage{xcolor}

\definecolor{dblue}{HTML}{143268}
\definecolor{dpink}{HTML}{B43283}

\usepackage{enumitem}

\renewcommand{\solutiontitle}{\noindent{\heiti\color{red}【解析】}\enspace}
\renewcommand{\questionlabel}{\color{dpink} \thequestion.}


\title{\huge \heiti \color{dblue} 每日一题(32)\\\pgfornament[scale=0.4]{84}}
\author{}
\date{}

\begin{document}

\maketitle

\thispagestyle{headandfoot}

% \qformat{\Large \textbf{题 \thequestion。}\hfill}

\begin{questions}

\setcounter{question}{31}

\question
% Origin: 2019 Mock AMC 9 #22, 2019 AMC 10 #25
设 $n$ 是一个合数,且不能整除 $(n-1)!$,求满足条件的 $n$ 的所有可能值。

\begin{solution}
    只需要满足存在一个质数 $p$,使得 $p^k\mid n$,但 $p^k\nmid (n-1)!$即可。
    设 $n=p^k\times q$,其中 $p$ 为质数,$k$、$q$ 为正整数。
    \begin{enumerate}[label={(\arabic*)}]
        \item 若 $q\geqslant 2$,则 $p^k\mid p^k(q-1) < n-1$,不满足条件,因此 $q=1$,$n=p^k$。由 $n$ 是合数知 $k\geqslant 2$。
        \item 若 $p>2$,则 $2p^{k-1}<p^k=n$,故$p^{k-1}\times 2p^{k-1}\mid (n-1)!$,而 $2(k-1)=k+(k-2)\geqslant k$,不满足条件,因此 $p=2$,$n=2^k$。
        \item 若 $k\geqslant 3$,则 $3\cdot 2^{k-2}<2^k=n$,故 $2^{k-2}\times \left(2\cdot 2^{k-2}\right)\times \left(3\cdot 2^{k-2}\right)\mid (n-1)!$,而 $3(k-2)=3k-6=k+(2k-6)\geqslant k$,不满足条件,因此 $k=2$,$n=4$。
    \end{enumerate}
    综上所述,$n=4$。
\end{solution}

\end{questions}

\end{document}
